%%%%%%%%%%%%%%%%%%%%%%%%%%%%%%%%%%%%%%%%% 
% Author: Sibi <sibi@psibi.in>
%%%%%%%%%%%%%%%%%%%%%%%%%%%%%%%%%%%%%%%%% 
\documentclass{article}
\usepackage{graphicx}
\usepackage{verbatim}
\usepackage{amsmath}
\usepackage{amsfonts}
\usepackage{amssymb}
\usepackage{tabularx}
\usepackage{mathtools}
\usepackage{braket}
\newcommand{\BigO}[1]{\ensuremath{\operatorname{O}\bigl(#1\bigr)}}
\setlength\parskip{\baselineskip}
\begin{document}
\title{Numbers and Sets}
\author{Sibi}
\date{\today}
\maketitle

% See here: http://tex.stackexchange.com/a/43009/69223
\DeclarePairedDelimiter\abs{\lvert}{\rvert}%
\DeclarePairedDelimiter\norm{\lVert}{\rVert}%

% Swap the definition of \abs* and \norm*, so that \abs
% and \norm resizes the size of the brackets, and the 
% starred version does not.
\makeatletter
\let\oldabs\abs
\def\abs{\@ifstar{\oldabs}{\oldabs*}}
% 
\let\oldnorm\norm
\def\norm{\@ifstar{\oldnorm}{\oldnorm*}}
\makeatother
\newpage

\section{Intervals}

\begin{align*}
  (a,b) = \set{ x: a < x < b}  \\
  [a,b] = \set{x : a <= x <= b } \\
  (a,b] = \set{x : a < x <= b } \\
  [a,b) = \set{x : a <= x < b } \\
\end{align*}

$(a,b)$ : open interval

$[a,b]$ : closed interval

$(a,b], [a,b)$ : Half open interval

\section{Absolute value}

% Ref: https://tex.stackexchange.com/a/47171/69223
\[
  \abs{x} = \begin{cases}
    x , & \text{if } x \geq 0 \\
    -x , & \text{if } x < 0 \\
  \end{cases}
\]

For any numbers x and y, the following statements are true:
\begin{itemize}
\item $\abs{x} < y$ iff $-y < x < y$
\item $\abs{x} \leq y$ iff $-y \leq x \leq y$
\item $\abs{x} < y$ iff either $x \leq -y$ or $x \geq y$
\item $\abs{x} < y$ iff either $x \leq -y$ or $x > y$
\item $\abs{xy} = \abs{x}\abs{y}$
\item $\text{If } y \neq 0 \text{ then } \abs{\frac{x}{y}} = \frac{\abs{x}}{\abs{y}}$
\end{itemize}

\section{Triangle Inequality}

For all numbers $x$ and $y$, $\abs{x + y} \leq \abs{x} + \abs{y}$.

\section{Exercise 1.1.6}

This already has a solution in the book but I would solve it in a slightly different way which IMO is more readable!

Solve $2\abs{x} - 3 >= \abs{x-1}$

There are two cases to consider:

% Todo: Add infinity symbol

\begin{itemize}
\item Case 1: $x >= 0$
  Now let us consider the sub-cases:
  \begin{enumerate}
  \item $x - 1 >= 0$

    From the original equality, we get $2x - 3 >= x - 1$. Solving it, we get $x >= 2$ which is equivalent to $[2, \inf)$
  \item $x - 1 < 0$

    From the original equality, we get $2x - 3 >= 1 - x$. Solving it, we get $x >= 4/3$. But this inequality is false for all values of $[0, 1)$.
  \end{enumerate}

\item  Case 2: $x < 0$
  Again, let us consider the sub-cases:
  \begin{enumerate}
  \item $x - 1 >= 0$

    This states that $x >= 1$. But this is a false case as $x < 0$.

  \item $x - 1 < 0$

    From the original equality, we get $-2x - 3 >= 1 - x$. Solving it,
    we get $x <= -4$ which is equivalent to ($(\-inf, -4]$). (See case
    1 in the book for better understanding.)
  \end{enumerate}
\end{itemize}

Combining them, we get:

$[2, \inf] \cup (\-inf, -\frac{2}{3}] \cup (\-inf, -4]$
which reduces to

$[2, \inf] \cup (\-inf, -4]$
\end{document}

