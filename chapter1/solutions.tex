%%%%%%%%%%%%%%%%%%%%%%%%%%%%%%%%%%%%%%%%% 
% Author: Sibi <sibi@psibi.in>
%%%%%%%%%%%%%%%%%%%%%%%%%%%%%%%%%%%%%%%%% 
\documentclass{article}
\usepackage{graphicx}
\usepackage{verbatim}
\usepackage{amsmath}
\usepackage{amsfonts}
\usepackage{amssymb}
\usepackage{tabularx}
\usepackage{mathtools}
\usepackage{braket}
\newcommand{\BigO}[1]{\ensuremath{\operatorname{O}\bigl(#1\bigr)}}
\setlength\parskip{\baselineskip}
\begin{document}
\title{Numbers and Sets}
\author{Sibi}
\date{\today}
\maketitle

% See here: http://tex.stackexchange.com/a/43009/69223
\DeclarePairedDelimiter\abs{\lvert}{\rvert}%
\DeclarePairedDelimiter\norm{\lVert}{\rVert}%

% Swap the definition of \abs* and \norm*, so that \abs
% and \norm resizes the size of the brackets, and the 
% starred version does not.
\makeatletter
\let\oldabs\abs
\def\abs{\@ifstar{\oldabs}{\oldabs*}}
% 
\let\oldnorm\norm
\def\norm{\@ifstar{\oldnorm}{\oldnorm*}}
\makeatother
\newpage

\section{Solution 1}

Let $x$ be an arbitrary number. Let us consider the cases:

\begin{itemize}
\item Case 1: $x - 5 >= 0$

  Solving $x - 5 >= 0$, we get $x >= 5$.
  
  From the original inequality, we get $x - 5 < 7$. Solving it, we get
  $x < 12$. The solution set is $[5, 12)$.
  
\item Case 2: $x - 5 < 0$

  Solving $x - 5 < 0$, we get $x < 5$.

  From the original inequality, we get $5 - x< 7$. Solving it, we get $x >
  -2$. The solution set is $(-2, 5)$.
\end{itemize}

Combining them, the solution set is $(-2, 5) \cup [5,12]$ which is
$(-2, 12)$.

\section{Solution 2}

Let $x$ be an arbitrary number. Let us consider the cases:

\begin{itemize}
\item Case 1: $4x + 2 >= 0$

  Solving $4x + 2 >= 0$, we get $ x >= -1/2$.
  
  From the original inequality, we get $4x + 2 <= 6$. Solving it, we get
  $x <= 1$. The solution set is $[-\frac{1}{2}, 1]$.
  
\item Case 2: $4x + 2 < 0$

  Solving $4x + 2 < 0$, we get $x < -1/2$.

  From the inequality, we get $-4x - 2 <= 6$. Solving it, we get
  $x >= -2$. The solution set is $[-2, -\frac{1}{2})$.
\end{itemize}

Combining them, the solution set is
$[-\frac{1}{2}, 1] \cup [-2, -\frac{1}{2})$ which is $[-2, 1]$.

\section{Solution 3}

Let $t$ be an arbitrary number. Let us consider the cases:

\begin{itemize}
\item Case 1: $5 - 2t >= 0$

  Solving $5 - 2t >= 0$, we get $t <= 2.5$

  From the original inequality, we get $5 - 2t >= 4$. Solving it, we
  get $t <= 1/2$. The solution set is $(-\infty, 1/2)$.

\item Case 2: $5 - 2t < 0$

  Solving $5 - 2t < 0$, we get $t > 2.5$.

  From the original inequality, we get $2t - 5 >= 4$. Solving it, we
  get $t >= 9/2$. The solution set is $[\frac{9}{2}, \infty]$.
\end{itemize}


Combining them, the solution set is $(-\infty, 1/2) \cup [\frac{9}{2},
  \infty]$.


\end{document}

