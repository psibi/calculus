%%%%%%%%%%%%%%%%%%%%%%%%%%%%%%%%%%%%%%%%% 
% Author: Sibi <sibi@psibi.in>
%%%%%%%%%%%%%%%%%%%%%%%%%%%%%%%%%%%%%%%%% 
\documentclass{article}
\usepackage{graphicx}
\usepackage{verbatim}
\usepackage{amsmath}
\usepackage{amsfonts}
\usepackage{amssymb}
\usepackage{tabularx}
\usepackage{mathtools}
\usepackage{braket}
\newcommand{\BigO}[1]{\ensuremath{\operatorname{O}\bigl(#1\bigr)}}
\setlength\parskip{\baselineskip}
\begin{document}
\title{Numbers and Sets}
\author{Sibi}
\date{\today}
\maketitle

% See here: http://tex.stackexchange.com/a/43009/69223
\DeclarePairedDelimiter\abs{\lvert}{\rvert}%
\DeclarePairedDelimiter\norm{\lVert}{\rVert}%

% Swap the definition of \abs* and \norm*, so that \abs
% and \norm resizes the size of the brackets, and the 
% starred version does not.
\makeatletter
\let\oldabs\abs
\def\abs{\@ifstar{\oldabs}{\oldabs*}}
% 
\let\oldnorm\norm
\def\norm{\@ifstar{\oldnorm}{\oldnorm*}}
\makeatother
\newpage

\section{Solution 1}

Let $x$ be an arbitrary number. Let us consider the cases:

\begin{itemize}
\item Case 1: $x - 5 >= 0$

  Solving $x - 5 >= 0$, we get $x >= 5$.
  
  From the original inequality, we get $x - 5 < 7$. Solving it, we get
  $x < 12$. The solution set is $[5, 12)$.
  
\item Case 2: $x - 5 < 0$

  Solving $x - 5 < 0$, we get $x < 5$.

  From the original inequality, we get $5 - x< 7$. Solving it, we get $x >
  -2$. The solution set is $(-2, 5)$.
\end{itemize}

Combining them, the solution set is $(-2, 5) \cup [5,12]$ which is
$(-2, 12)$.

\section{Solution 2}

Let $x$ be an arbitrary number. Let us consider the cases:

\begin{itemize}
\item Case 1: $4x + 2 >= 0$

  Solving $4x + 2 >= 0$, we get $ x >= -1/2$.
  
  From the original inequality, we get $4x + 2 <= 6$. Solving it, we get
  $x <= 1$. The solution set is $[-\frac{1}{2}, 1]$.
  
\item Case 2: $4x + 2 < 0$

  Solving $4x + 2 < 0$, we get $x < -1/2$.

  From the inequality, we get $-4x - 2 <= 6$. Solving it, we get
  $x >= -2$. The solution set is $[-2, -\frac{1}{2})$.
\end{itemize}

Combining them, the solution set is
$[-\frac{1}{2}, 1] \cup [-2, -\frac{1}{2})$ which is $[-2, 1]$.

\section{Solution 3}

Let $t$ be an arbitrary number. Let us consider the cases:

\begin{itemize}
\item Case 1: $5 - 2t >= 0$

  Solving $5 - 2t >= 0$, we get $t <= 2.5$

  From the original inequality, we get $5 - 2t >= 4$. Solving it, we
  get $t <= 1/2$. The solution set is $(-\infty, 1/2)$.

\item Case 2: $5 - 2t < 0$

  Solving $5 - 2t < 0$, we get $t > 2.5$.

  From the original inequality, we get $2t - 5 >= 4$. Solving it, we
  get $t >= 9/2$. The solution set is $[\frac{9}{2}, \infty]$.
\end{itemize}


Combining them, the solution set is $(-\infty, 1/2) \cup [\frac{9}{2},
\infty]$.

\section{Solution 4}

Let $x$ be an arbitrary number. Let us consider the cases:

\begin{itemize}
\item Case 1: $x - 4 >= 0$

  Solving $x - 4 >= 0$, we get $x >= 4$

  From the original inequality, we get $x - 4 < x$. Solving it, we
  get $0 < 4$ which is true. The solution set is $[4, \infty)$.

\item Case 2: $x - 4 < 0$

  Solving $x - 4 < 0$, we get $x < 4$.

  From the original inequality, we get $4 - x < x$. Solving it, we
  get $x > 2$. The solution set is $(2, 4)$.
\end{itemize}

Combining them, the solution set is $(2, 4) \cup [4, \infty)$ which
gets reduced to $(2,\infty)$.

\section{Solution 5}

Let $x$ be an arbitrary number. Let us consider the cases:

\begin{itemize}
\item Case 1: $x - 4 >= 0$

  Solving $x - 4 >= 0$, we get $x >= 4$

  From the original inequality, we get $2x - 8 < x$. Solving it, we
  get $x < 8$ which is true. The solution set is $[4, 8)$.

\item Case 2: $x - 4 < 0$

  Solving $x - 4 < 0$, we get $x < 4$.

  From the original inequality, we get $-2(x - 4) < x$. Solving it, we
  get $x > \frac{8}{3}$. The solution set is $(\frac{8}{3}, 4)$.
\end{itemize}

Combining them, the solution set is $(\frac{8}{3}, 4) \cup [4, 8)$ which
gets reduced to $(\frac{8}{3}, 8)$.

\section{Solution 6}

Let $x$ be an arbitrary number. Let us consider the cases:

\begin{itemize}
\item Case 1: $x + 4 >= 0$

  Solving $x + 4 >= 0$, we get $x >= -4$

  From the original inequality, we get $x + 4 < x$. Solving it, we
  get $4 < 0$ which is false.

\item Case 2: $x + 4 < 0$

  Solving $x + 4 < 0$, we get $x < -4$.

  From the original inequality, we get $-x - 4 < x$. Solving it, we
  get $x > -2$. But this is false for all values of $(-\infty, -4))$.
\end{itemize}

This inequality doesn't contain any solution.

\section{Solution 7}

Let $x$ be an arbitrary number. Let us consider the cases:

\begin{itemize}
\item Case 1: $x + 4 >= 0$

  Solving $x + 4 >= 0$, we get $x >= -4$

  From the original inequality, we get $x + 8 < 2x$. Solving it, we
  get $x > 4$. The solution set is $(4, \infty)$.

\item Case 2: $x + 4 < 0$

  Solving $x + 4 < 0$, we get $x < -4$.

  From the original inequality, we get $-x - 4 < 2x$. Solving it, we
  get $x > -\frac{4}{3}$. But this is false for all values of
  $(-\infty, -4)$. 
\end{itemize}

Combining them, the solution set is $(4, \infty)$.

\section{Solution 8}

Let $u$ be an arbitrary number. Let us consider the cases:

\begin{itemize}
\item Case 1: $6 - 2u >= 0$

  Solving $6 - 2u >= 0$, we get $u <= 3$

  From the original inequality, we get $6 - 2u - 3 < u$. Solving it, we
  get $u > 1$. The solution set is $(1, 3]$.

\item Case 2: $6 - 2u < 0$

  Solving $6 - 2u < 0$, we get $u > 3$.

  From the original inequality, we get $-(6 - 2u) - 3 < u$. Solving it, we
  get $u < 9$. The solution set is $(3, 9)$.
\end{itemize}

Combining them, the solution set is $(1, 9)$.

\section{Solution 9}

Let $y$ be an arbitrary number. Let us consider the cases:

\begin{itemize}
\item Case 1: $y >= 0$

  From the original inequality, we get $3y <= y + 10$. Solving it, we
  get $y <= 5$. The solution set is $[0, 5]$.

\item Case 2: $y < 0$

  From the original inequality, we get $-(3y) <= -y + 10$. Solving it, we
  get $y >= -5$. The solution set is $[-5, 0)$.
\end{itemize}

Combining them, the solution set is $[-5, 5]$.

\section{Solution 10}

Let $x$ be an arbitrary number. Let us consider the cases:

\begin{itemize}
\item Case 1: $x >= 0$

  \begin{enumerate}
  \item $x - 3 >= 0$

    Solving $x - 3 >= 0$, we get $x >= 3$. Solving for the original
    inequality, we get $ 0 >= 2$ which is false.
    
  \item $x - 3 < 0$

    Solving $x - 3 < 0$, we get $x < 3$. Solving for the original
    inequality, we get $-x + 3 >= x - 1$ which reduces to $x <= 2$.
    The solution set is $[0, 2]$.
  \end{enumerate}

\item Case 2: $x < 0$

  \begin{enumerate}
  \item $x - 3 >= 0$

    Solving $x - 3 >= 0$, we get $x >= 3$. But this is false since
    $x < 0$.
    
  \item $x - 3 < 0$

    Solving $x - 3 < 0$, we get $x < 3$. Solving for the original
    inequality, we get $-x + 3 >= -x - 1$ which reduces to $3 >= -1$.
    The solution set is $(-\infty, 0)$.
  \end{enumerate}

\end{itemize}

Combining them, the solution set is $(-\infty, 2]$.

\section{Solution 11}

Let $x$ be an arbitrary number. Let us consider the cases:

\begin{itemize}
\item Case 1: $x >= 0$

  \begin{enumerate}
  \item $2x - 3 >= 0$

    Solving $2x - 3 >= 0$, we get $x >= 1.5$. Solving for the original
    inequality, we get $ x >= 2$. The solution set is $[2, \infty)$
    
  \item $2x - 3 < 0$

    Solving $2x - 3 < 0$, we get $x < 1.5$. Solving for the original
    inequality, we get $x <= 4/3$. The solution set is $[0, 4/3]$.
  \end{enumerate}

\item Case 2: $x < 0$

  \begin{enumerate}
  \item $2x - 3 >= 0$

    Solving $2x - 3 >= 0$, we get $x >= 1.5$. But this is false since
    $x < 0$.
    
  \item $2x - 3 < 0$

    Solving $2x - 3 < 0$, we get $x < 1.5$. Solving for the original
    inequality, we get $-2x + 3 >= -x - 1$ which reduces to $x <= 4$.
    The solution set is $(-\infty, 0)$.
  \end{enumerate}

\end{itemize}

Combining them, the solution set is
$[2, \infty) \cup [0, 4/3] \cup (-\infty, 0)$ which reduces to
$(-\infty, 4/3] \cup [2, \infty)$.

\section{Solution 12}

Let $x$ be an arbitrary number. Let us consider the cases:

\begin{itemize}
\item Case 1: $x >= 0$

  \begin{enumerate}
  \item $2x - 2 >= 0$

    Solving $2x - 2 >= 0$, we get $x >= 1$. Solving for the original
    inequality, we get $ x > 1$. The solution set is $(1, \infty)$
    
  \item $2x - 2 < 0$

    Solving $2x - 2 < 0$, we get $x < 1$. Solving for the original
    inequality, we get $x < 1$. The solution set is $[0, 1)$.
  \end{enumerate}

\item Case 2: $x < 0$

  \begin{enumerate}
  \item $2x - 2 >= 0$

    Solving $2x - 2 >= 0$, we get $x >= 1$. But this is false since
    $x < 0$.
    
  \item $2x - 2 < 0$

    Solving $2x - 2 < 0$, we get $x < 1$. Solving for the original
    inequality, we get $2 - 2x > -x - 1$ which reduces to $x <= 3$.
    The solution set is $(-\infty, 0)$.
  \end{enumerate}

\end{itemize}

Combining them, the solution set is
$[0, 1) \cup (1, \infty) \cup (-\infty, 0)$ which reduces to
$(-\infty, 1) \cup [1, \infty)$.

\section{Solution 13}

Let $x$ be an arbitrary number. Let us consider the cases:

\begin{itemize}
\item Case 1: $3x - 6 >= 0$

  Solving $3x - 6 >= 0$, we get $x >= 2$.
  
  \begin{enumerate}
  \item $3 - 6x >= 0$

    Solving $3 - 6x >= 0$, we get $x <= 0.5$. But this is false for
    $x >= 2$.
    
  \item $3 - 6x < 0$

    Solving $3 - 6x < 0$, we get $x > 0.5$. Solving for the original
    inequality, we get $x >= -1$. The solution set is $[2, \infty)$.
  \end{enumerate}

\item Case 2: $3x - 6 < 0$

  Solving $3x - 6 < 0$, we get $x < 2$.

  \begin{enumerate}
  \item $3 - 6x >= 0$

    Solving $3 - 6x >= 0$, we get $x <= 0.5$. Solving for the original
    inequality, we get $x <= -1$. The solution set is $(-\infty, -1]$.
    
  \item $3 - 6x < 0$

    Solving $3 - 6x < 0$, we get $x > 0.5$. Solving for the original
    inequality, we get $6 -3x <= 6x - 3$ which reduces to $x >= 1$.
    The solution set is $[1, 2)$.
  \end{enumerate}

\end{itemize}

Combining them, the solution set is
$[2, \infty) \cup (-\infty, -1] \cup [1, 2)$ which reduces to
$[1, \infty) \cup (-\infty, -1]$.

\end{document}

